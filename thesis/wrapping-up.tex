\section{Conclusion}\label{sec:conclusion}
This thesis managed to train neural networks with 100\% accuracy in the single stack version of Nim for stack sizes up to 1000.
The success was mostly due to the realization that the output should be encoded directly instead of using one-hot output encoding.
Although one-hot encoding keeps the output in a safe bound, it finally makes it harder for the NNs to find the correct structure and weights in the case of Nim.
Other parameter configurations were explored, however, most of them showed to be similarly effective as the initial configuration, except for the weight reduction of topology increasing mutations, which demonstrated to be effective.
\\ \\
In the full Nim game, the NNs were able to reach an accuracy of 100\% for two stacks with two matches each and reached an accuracy of ~75\% for two stacks with 4 matches each.
The NNs couldn't adapt to playing the perfect move for all 24 possible game states in the latter game and got stuck in local minimum where the NNs of a population didn't predict the perfect move for all 24 possible game states.
\\\\
A plausible explanation for the finding that the NNs could not be trained to full accuracy in the more complex configuration might lie in the game which was used as benchmark for the algorithm.
The Nim game might be a tough challenge for NNs to solve as the algorithm to play Nim perfectly is not easily translatable into neurons.
If this is the case, this thesis demonstrates the ability of ENN algorithms to adapt to complex tasks to a significant extent nevertheless.
That said, this implementation of ENNs leaves room for future development (detailed in~\ref{sec:future-work}), which might also increase the performance of the algorithm in the Nim game.
%There are two explanations for the finding that the NNs did not become fully accurate in the more complex configuration.
%Either, the Nim game is an unsuited problem for NNs to solve as the algorithm to play Nim perfectly is not easily translatable into neurons, or this implementation and or parameters of the ENN need refinement.
%To tell which of the two reasons apply, more research needs to be conducted in this field, which will be discussed further in the last section (\ref{sec:future-work}).
\\ \\
This thesis sometimes dealt with varying results for tests within the same parameter configuration.
In an application, it is therefore advisable to train several populations on the same problem to then choose the best performing one.

\section{Auto Review}\label{sec:auto-review}
This research was a large journey for me to realize, which finally resulted in a ENN framework that I proved to be able to learn functions without human supervision.
The concept of Neural Networks evolving on their own and finally acquiring some sort of skill seemed very theoretical and difficult for me, which is why it is such a delight for me to actually have created a practical implementation of ENNs that is able to learn a game on its own.
\\ \\
When reflecting on the planing process, I definitely underestimated the amount of time needed for the development and particularly the testing process.
Even though I tried to optimize the program with algorithm design choices and parallel processing, the learning process still demanded a lot of time.
This limited the amount of tests this thesis was able to evaluate.
%I also discovered a few bugs during the testing time, which forced me to redo the testing.
%I planned to be done with a beta program many months before the thesis end to then have time to test and refine my program.
%However, I finished the program much later and found many bugs which forced me to repeat the whole testing.
%The final tests were therefore done in only a few days, which left me with little room to search for optimal parameters.
\\
Another thing I realized is that the scientific method of testing is essential.
Mmy approach of starting with a simple solution and then testing different parameters and features one at a time proved to be very effective.
It helped me find different, more effective configurations than the ones I had previously imagined to be plausible.
\\
In terms of implementation, I had some struggles with the parameters being defined all over the place in the project.
I would therefore create a struct that handles all parameters and also implement automized testing which would allow me to run all tests without supervision.
\\
In general, my object-oriented programming approach worked fine and my project structure made the codebase easily understandable and editable throughout my research.
%Still, more consistent commenting and documentation would have further improved clarity of the code.


\section{Future Work}\label{sec:future-work}
As already mentioned, this thesis leaves many possibilities for future research, which might also answer whether the Nim game can be learned by ENN algorithms.
In specific, future work might include the following:

\subsubsection{More Testing}
Because of the time constraints, this research tested a number of configurations with limited tests each.
More test runs within a set of parameters will therefore increase statistical significance of the results and more tests with different sets of parameters might help finding better solutions.
This thesis has already tested the effect of some modifications of the fitness exponent, new population composition, and population size parameters without finding consistent evidence for significant improvement of performance with any of these parameters.
In-depth testing of the effect of these parameters would facilitate their fine-tuning and might finally improve the ENN learning process.
In specific, following parameters might be particularly important to examine:
\begin{itemize}
    \item \textbf{Fitness exponent and new population composition:} These parameters influence what agents make up each population and are therefore critical to optimize to ensure the algorithms success.
    \item \textbf{Population size:} The approach of this thesis was to have a relatively small population too keep the needed computation per generation low, enlarging the population however, might allow for more diversity within a population which might help new strategies to evolve.
    \item \textbf{Mutation rate:} This thesis kept the rate of mutation minimal to ensure consistent development, however, increasing the rate of mutations might accelerate innovation of NNs.
    \item \textbf{Mutation types:} During the testing process, the weight of the topology-increasing mutation were decreased, which resulted in significantly less unnecessary complexity of the NNs.
    Tuning the weights of other mutations might help prioritize the important mutations and therefore increase the efficiency of the whole algorithm.
\end{itemize}
Evolutionary computation could also be employed to optimize different parameters of the ENN algorithm itself and therefore improve the ENN learning process.

\subsubsection{Faster Computation}
To enable more tests, the computational performance of the ENN algorithm could to be optimized.
This can be done by converting the NNs into a weight matrix and then compute its predictions with hardware accelerated vector matrix multiplications.
However, the complex topologies of the NNs in this approach make it harder to do so than with other NNs.
Also many small performance improvements can still be done in this project which might cumulatively make the whole algorithm drastically faster.

\subsubsection{Algorithm Design}
There is a large variety of additional features that might improve the performance of the ENN algorithm.
For example, the NEAT algorithm, which served as initial inspiration, also uses genetic encoding, tracks genes using historical markings and protects innovation using speciation\footcite{Neat_02}.
In their tests, the developers of the NEAT algorithm were able to prove these features to effectively improve the results of the learning process.
These features might therefore also improve the performance of the ENN algorithm in the Nim game.

\section{Remark: Tools used during Research}
To complete this research, following tools were used:
\begin{itemize}
    \item \textbf{Rust} as the programming language\footcite{rust23}.
    \item \textbf{JetBrains RustRover} as the integrated development environment\footcite{rustrover}.
    \item \textbf{Git} as the version control system\footcite{git}.
    \item \textbf{GitHub} as the online host of the repository\footcite{github}.
    \item \textbf{GitHub Copilot} as the AI code completion tool to accelerate programming\footcite{github_copilot2021}.
    \item \textbf{LaTex} as the typesetting system for the thesis\footcite{latex}.
    \item \textbf{OpenAI ChatGPT} to enhance the language for some pre-written text parts and help troubleshoot technical problems\footcite{chatgpt}.
    \item \textbf{Phoenix GUI} as a graphing tool to visualize the results of the tests\footcite{phoenixgui}.
\end{itemize}
A big thank goes also to Gerd Altmann, for providing the cover image of this thesis\footcite{pixabay_neuronen}.
\cleardoublepage % Ensures proper placement in book format
\phantomsection % For hyperref and correct PDF bookmarks
\addcontentsline{toc}{chapter}{References} % Adds to the Table of Contents
\printbibliography[title=References]
\\
\vspace*{3em}
\setlength{\fboxsep}{10pt} % Abstand zwischen Text und Rahmen (Standard ist 3pt)
\noindent
\fbox{\parbox{\dimexpr\textwidth-2\fboxsep-2\fboxrule}
{Ich, Lucien Kissling, 6e, erkläre hiermit, dass ich die vorliegende Arbeit «Evolutionary Neural Networks: Designing a NEAT-Inspired Algorithm for Strategic Game Learning» selbständig und ohne Benützung anderer als der angegebenen Quellen oder Hilfsmittel verfasst bzw. gestaltet habe.
\vspace{1cm}
\hspace{-0.3cm}
\\\\
Ort, Datum: \underline{\hspace{5cm}} \hfill Unterschrift: \underline{\hspace{5cm}}}}
